\documentclass[../main.tex]{subfiles}
\graphicspath{{\subfix{../images/}}}

\begin{document}

\section{Finite fields}

We want the alphabet $\Sigma$ we use to have the structure of a field such as $\R$ or $\C$ to be able to describe codes as vector spaces over the alphabet. Furthermore, in many communication channels the alphabets are finite, so we want a finite set with the structure of a field. The main reference for this section is \cite[Chapter 3]{ling2004coding}. See also the lecture notes on finite fields in \cite{kopparty2023topics}.

\subsection{Field theory}

A \emph{field} is a commutative ring $F$ with $F^* = F \setminus \{0\}$. This means that the usual rules of arithmetic in fields such as $\Q, \R$ or $\C$ apply. A \emph{finite field} is a field with finitely many elements.

\begin{example}
Consider the ring of integers modulo $m$, \ie, $\Z_m = \Z / m\Z$. If $m = pq$ is a composite number with $p, q > 1$, then $p, q \not\equiv 0 \pmod m$, but $pq = m \equiv 0 \pmod m$. Therefore, $\Z_m$ is not a field, since it has zero divisors. On the other hand, if $p$ is a prime and $0 < r < p$, then there are integers $a, b \in \Z$ such that
\begin{equation*}
    ap + br = 1
\end{equation*}
by Bezout's identity, so $br \equiv 1 \pmod p$. Thus, $r$ has an inverse element $b = r^{-1}$ in $\Z_p$ if $r \not\equiv 0 \pmod p$. The ring $\Z_m$ is a field if and only if $m$ is prime.
\end{example}

The \emph{characteristic} of a field is the smallest positive integer $c$ such that
\begin{equation*}
    \underbrace{1_F + 1_F + \cdots + 1_F}_{\text{$c$ terms}} = c \cdot 1_F = 0_F
\end{equation*}
if it exists, otherwise the characteristic is zero. We write, $\chr(F) = c$ or $\chr(F) = 0$. Recall that the characteristic of a field is either 0 or a prime $p$. If $\chr(F) = 0$, then $F$ has infinitely many elements (it contains a copy of $\Q$), so a finite field has characteristic $p > 0$. The characteristic of the field $\Z_p$ is $p$.

Let $F$ be a field with characteristic $\chr(F) = p$. Then, it is easy to verify that
\begin{align*}
    \Z_p \to F, \quad [n] \mapsto n \cdot 1_F
\end{align*}
is an injective homomorphism. Therefore, $F$ contains a copy of $\Z_p$ as a subfield. If $F$ is a finite field, then $F$ is a vector space of finite dimension $n$ over $\Z_p$. Thus, $\lvert F \rvert = p^n$ for some $n$ and $p = \chr(F)$. We can deduce that the finite field of size $p$ is unique up to isomorphism and it is denoted by $\F_p$ or $\operatorname{GF}(p)$ (GF stands for Galois field). The fields $\F_p$ for all primes $p$, as well as $\Q$, are said to be \emph{prime fields} and every field contains a unique copy of a prime field that is determined by the characteristic.

\begin{example}
The smallest field is the field with 2 elements: $\F_2 = \Z_2 = \{0, 1\}$. This field is known as the \emph{binary field} and it is popular due to its simple structure and ease of computer implementation. The field with three elements is known as the \emph{ternary field}: $\F_3 = \Z_3 = \{0, 1, 2\}$.

There is a finite field of size $4 = 2^2$, but it is not isomorphic $\Z_4$, which is not a field as $4$ is composite.

There is no finite field of size $6 = 2 \cdot 3$, since $6$ is not a prime power.
\end{example}

\begin{example}
Every finite integral domain $F$ is a field. Consider $x \neq 0 \in F$. Then, $xy = xy'$ if and only if $y = y'$ due to the fact that $F$ is an integral domain ($x(y - y') = 0$). Thus, $\{xy \mid y \in F \setminus \{0\} \}$ contains $\lvert F \rvert - 1$ nonzero elements, so one of them must be $1_F$, \ie, every nonzero element has an inverse.
\end{example}

\subsection{Polynomial rings}

Recall that $F[x]$ denotes the ring of univariate polynomials in $x$ with coefficients in $F$. A polynomial is said to be \emph{monic} if its leading coefficient is 1. A polynomial $f$ (also denoted with $f(x)$) is \emph{irreducible} if it cannot be expressed as $f = gh$ for some $g, h \in F[x]$ with $\deg(g), \deg(h) < \deg(f)$. If $f_1, \dots, f_t \in F[x]$ are polynomials, then $I = (f_1, \dots, f_t) \subseteq F[x]$ is the \emph{ideal} generated by these polynomials.

Let $f, g \in F[x]$ and $g \neq 0$. By the division algorithm there exists unique polynomials $s, r \in F[x]$ such that
\begin{equation*}
    f = sg + r, \quad \text{and} \quad \deg(r) < \deg(g)~\text{or}~r = 0.
\end{equation*}

\begin{example}
Let $g = x - \alpha \in F[x]$ for some $\alpha \in F$. Then
\begin{equation*}
    f = sg + r
\end{equation*}
where $\deg(r) < \deg(g) = 1$, \ie, $r$ is a constant. By evaluating at $x = \alpha$ we get that $r = f(\alpha)$ as $g(\alpha) = 0$. If $\alpha$ is a root of $f$, then $f = s \cdot (x - \alpha)$. Therefore, a polynomial of degree $d$ over a field has at most $d$ roots.
\end{example}

Due to the division algorithm, all ideals in $F[x]$ are principal (every element in the ideal is a multiple of the nonzero element of lowest degree).

\begin{lemma}\label{lem:quotient_by_irreducible_polynomial}
Let $f \in F[x]$ be an irreducible polynomial. Then, $F[x] / (f)$ is a field of extension degree $\deg(f)$ over $F$.
\end{lemma}

\begin{proof}
It is clear that $F[x] / (f)$ is a commutative ring, so we need to first show that it is a field. Assume without loss of generality that $f$ is monic. Let $g \in F[x]$ be a polynomial and consider the ideal $I = (f, g)$ that is generated by $f$ and $g$. As all ideals are principal, we find that $I = (h$ for some monic $h \in F[x]$. Thus, $f$ is a multiple of $h$. By the irreducibility of $f$, we have that either $h = f$ or $h = 1$. In the first case $g$ is a multiple of $h = f$, so $g \equiv 0 \pmod f$. In the second case, $1 \in I$, so there are polynomials $p, q \in F[x]$ such that $1 = pf + qg$. Therefore, $qg \equiv 1 \pmod f$. Hence, all nonzero elements in $F[x] / (f)$ have an inverse.

Let $n = \deg(f)$. Then, the elements $1, x, \dots, x^{n-1}$ span $F[x] / (f)$, since any polynomial $g \in F[x]$ is equivalent to a polynomial of degree $< n$ modulo $f$. Furthermore,
\begin{equation*}
    p = a_0 + a_1x + \dots + a_{n-1}x^{n-1} \equiv 0 \pmod f \implies p = 0
\end{equation*}
as the only multiple of $f$ that has degree $< n$ is the zero polynomial. Hence, $1, x, \dots, x^{n-1}$ are linearly independent. Therefore, $F[x] / (f)$ has extension degree $n$ over $F$.
\end{proof}

With the above lemma we may construct finite fields by starting with irreducible polynomials over $\F_p$. If $f \in \F_p[x]$ is irreducible of degree $n$, then $\F_p[x] / (f)$ is a finite field of size $p^n$. In fact, this is essentially the same way that we constructed the field $\Z_p = \Z / p\Z$, since both $p\Z$ and $(f)$ are maximal ideals in the rings $\Z$ and $F[x]$, respectively.

\begin{example}
The polynomial $f = x^2 + x + 1 \in \F_2[x]$ is irreducible, since it has no roots in $\F_2$. Therefore, $\F_2[x] / (f)$ is a finite field of size $2^2 = 4$ and it consists of the elements $0, 1, x, x + 1$. The addition in this field is carried out by regular polynomial addition modulo 2, while multiplication is carried out modulo the polynomial $f$.
\end{example}

\begin{example}
The polynomial $x^2 + 1 \in \R[x]$ is irreducible, but $x^2 + 1 \in \F_2[x]$ is not. You have to be careful when showing if a polynomial is irreducible over a specific field.
\end{example}

\subsection{Structure of finite fields}

Let $F$ be a finite field with $q = p^n$ elements. Recall that $F^*$ is a group with $q - 1$ elements. Thus, by Lagrange's theorem, $\beta^{q - 1} = 1$ for all $\beta \in F^*$. Thus, $\beta^q = \beta$ for all $\beta \in F$.

\begin{lemma}\label{lem:unique_subfield}
Let $F \subset E$ be fields with $\lvert F \rvert = q$. Then,
\begin{equation*}
    F = \{\beta \in E \mid \beta^q = \beta \}.
\end{equation*}
\end{lemma}

\begin{proof}
Clearly, $F$ is contained in the set on the right hand side. Furthermore, the polynomial $x^q - x$ can have at most $q$ roots, which gives the equality.
\end{proof}

\begin{lemma}\label{lem:Frobenius_automorphism}
Let $F$ be a finite field with $\chr(F) = p$. Then, $\sigma \colon F \to F$, $\sigma(x) = x^p$ is an automorphism of $F$.
\end{lemma}

\begin{proof}
It is clear that $\sigma(xy) = (xy)^p = x^p y^p = \sigma(x)\sigma(y)$. Furthermore,
\begin{equation*}
    \sigma(x + y) = (x + y)^p = \sum_{i=0}^p \binom{p}{i}x^i y^{p-i} = x^p + y^p = \sigma(x) + \sigma(y,)
\end{equation*}
since $\binom{p}{i} \equiv 0 \pmod p$ for $0 < i < p$.

As noted above, $\sigma(x) = x$ for all $x \in \F_p \subseteq F$. Therefore, $\sigma$ is a linear map over the subfield $\F_p$. Hence, $\sigma$ is bijective if and only if $\ker(\sigma) = \{0\}$. We see that $\sigma(x) = x^p = 0$ implies that $x = 0$, so $\sigma$ is an isomorphism.
\end{proof}

The automorphism $\sigma$ defined above is called the \emph{Frobenius automorphism}.

\begin{theorem}
There is a unique finite field of size $p^n$ up to isomorphism.
\end{theorem}

\begin{proof}[Proof sketch]
The finite field of size $q = p^n$ is the splitting field of the polynomial $x^q - x$, so is unique up to isomorphism.
\end{proof}

As finite fields are unique up to isomorphism, we denote the finite field of size $q = p^n$ for prime $p$ and $n \geq 1$ by $\F_q$ (or $\operatorname{GF}(q)$).

Recall that the \emph{order} of a nonzero element $g \in \F_q^*$ is the smallest positive integer $m$ such that $g^m = 1$, denoted by $\ord(g) = m$. The order of an element equals the size of the (multiplicative) group generated by that element.

\begin{theorem}
The group of units $\F_q^*$ is a cyclic group.
\end{theorem}

\begin{proof}
Let $G = \F_q^*$ and $n = \lvert G \rvert$. For each divisor $d$ of $n$, define
\begin{equation*}
    G_d = \{ g \in G \mid \ord(g) = d \} \subseteq \{x \in G \mid x^d - 1 = 0 \}.
\end{equation*}
Recall that the order of an element divides the order of the group by Lagrange's theorem.

If $G_d \neq \emptyset$, let $g \in G_d$. Then,
\begin{align*}
    \langle g \rangle = \{x \in G \mid x^d - 1 = 0 \}.
\end{align*}
since $\lvert \langle g \rangle \rvert = d$ and $\lvert \{x \in G \mid x^d - 1 = 0\} \rvert \leq d$ as the polynomial $x^d - 1$ can have at most $d$ roots. Therefore, $G_d$ is the set of generators of $\langle g \rangle$ and has size $\varphi(d)$, where $\varphi$ denotes Euler's totient function.

Finally,
\begin{equation*}
    n = \lvert G \rvert = \sum_{d \mid n} \lvert G_d \rvert \leq \sum_{d \mid n} \varphi(d) = n.
\end{equation*}
To achieve equality, we must have that $\lvert G_d \rvert = \varphi(d)$ for all $d \mid n$. In particular, $\lvert G_n \rvert = \varphi(n) > 0$. Thus, $G$ contains an element of order $n$.
\end{proof}

If $\gamma \in \F_q$ has order $q - 1$, then $F = \{0, 1, \gamma, \gamma^2, \dots, \gamma^{q - 2}\}$. Such a $\gamma$ is said to be a \emph{primitive element}. As $\F_q$ is cyclic, primitive elements must exist, but are not in general unique.

\end{document}