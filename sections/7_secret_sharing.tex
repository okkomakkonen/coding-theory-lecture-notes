\documentclass[../main.tex]{subfiles}
\graphicspath{{\subfix{../images/}}}

\begin{document}

\section{Secret sharing}

In this section we will discuss one particular application of coding theory to data security and cryptography, namely secret sharing. Secret sharing was introduced independently by Shamir and Blakley in 1979, and has since seen a lot of attention from the cryptography and coding theory communities. For a general reference to secret sharing, see \cite{padro2012lecture}. For the linear secret sharing section we follow the presentation in \cite{chen2007secure}.

\subsection{Perfect security}

An encryption scheme turns a message $m$ to a ciphertext $c$ by a random transformation. The randomness of the encryption process may come from a key that can be used to invert the process and obtain the original message from the ciphertext. There are many definitions for the security of such schemes, but we will consider the strongest possible form of security. An encryption scheme is said to have \emph{perfect security} (or \emph{information-theoretic security}) if seeing the ciphertext yields no additional information about the message. One way to state this is that
\begin{align*}
    \P[\texttt{ciphertext} = c \mid \texttt{message} = m] = \P[\texttt{ciphertext} = c \mid \texttt{message} = m']
\end{align*}
for all possible messages $m, m'$ and ciphertext $c$. The probabilities are taken over the randomness in the encryption process.

One way to achieve perfect security is by \emph{one-time padding}, where the message is encrypted by adding a random \emph{mask} to the message. In particular, let $m$ be a message in some finite additive group $G$ and choose the noise $r \in G$ uniformly at random independently from $m$. Then, let $c = m + r$ be the ciphertext. Clearly, having $c$ and $r$ allows decoding of $m = c - r$. For all $r \in G$ we have that $\P[\texttt{noise} = r] = \tfrac{1}{\lvert G \rvert}$. Let $m, m' \in G$ be two messages. Then,
\begin{align*}
    \P[\texttt{ciphertext} = c \mid \texttt{message} = m] &= \P[\texttt{noise} = c - m] \\
    &= \P[\texttt{noise} = c - m'] \\
    &= \P[\texttt{ciphertext} = c \mid \texttt{message} = m'].
\end{align*}
Therefore, the one-time pad encryption method has perfect security. The group is assumed to be finite for the existence of the uniform distribution.

\subsection{Introduction to secret sharing}

Secret sharing is a cryptographic method of \emph{sharing} a secret value to a set of $n$ participants such that only some \emph{admissible sets} of participants may reveal the secret, while some \emph{forbidden sets} of participants will not learn anything about the secret. A common choice of the admissible and forbidden sets is determined by a threshold $t$, meaning that any set of participants of size $\leq t$ should not be able to reveal the secret, while any set of participants of size $> t$ will be able to reveal the secret. Such a secret sharing system is known as \emph{threshold secret sharing}. Notice that the admissible sets do not need any additional information, such as an encryption key, in addition to their shares for reconstruction.

Secret sharing has been used to store important cryptographic keys, such as the DNSSEC root keys, which are needed to reset some certificates on the internet\footfullcite{popsci2010order}. Secret sharing is advantageous, since these keys are vitally important, so not all of the shares should be needed for reconstruction. On the other hand, sufficiently many of the shares should be needed for reconstruction to not compromise the security of the key. The DNSSEC system uses seven shares, where any five can be used to reconstruct the root key.

\begin{example}[Shamir secret sharing]
Let $s \in \F_q$ be the secret value and choose $f \in \F_q[x]^{< t}$ uniformly at random. Denote $\hat{f} = s + x \cdot f$ the random polynomial of degree $\leq t$ with the property that $\hat{f}(0) = s$. Let $\alpha_1, \dots, \alpha_n \in \F_q$ be distinct nonzero values and share $\hat{f}(\alpha_1), \dots, \hat{f}(\alpha_n)$ with the $n$ participants.

As $\deg(\hat{f}) \leq t$, any $t + 1$ of the $n$ participants are able to interpolate the polynomial $\hat{f}$ and reveal $s = \hat{f}(0)$. Further, consider a set of $t$ participants $i_1, \dots, i_t \in [n]$. Let $s \in \F_q$. Then there is a unique polynomial of degree $\leq t$ running through the points $(0, s), (\alpha_{i_1}, \hat{f}(\alpha_{i_1})), \dots, (\alpha_{i_t}, \hat{f}(\alpha_{i_t}))$. This means that \emph{any} secret value is possible given the shares held by the participants.
\end{example}

\subsection{Linear secret sharing}

There are many different (and somewhat equivalent) ways of describing general secret sharing schemes. We will follow the presentation in \cite[Section 4.2]{chen2007secure} as it fits nicely within the coding theory topics that we have discussed previously.

We may build a linear secret sharing scheme by choosing two linear codes $\mathcal{S}, \mathcal{C} \subseteq \F_q^n$. Let us denote $\hat{\mathcal{C}} = \mathcal{S} + \mathcal{C}$. The secret value is $s \in \mathcal{S}$, which means that there are a total of $q^{\dim(\mathcal{S})}$ possible secrets. Furthermore, we choose $c \in \mathcal{C}$ uniformly at random and set $\hat{c} = s + c$. The $n$ participants are given one coordinate of $\hat{c}$ each. This looks a bit like the one-time pad encryption scheme, except that the noise is not chosen uniformly at random from the ambient space $\F_q^n$, but rather from the subspace $\mathcal{C}$. However, the security of secret sharing comes from the fact that the noise will still look uniform in the ambient space when we restrict to some subset of the coordinates.

Given $\hat{c} \in \hat{\mathcal{C}}$ we may uniquely decode $s$ if and only if $\mathcal{S} \cap \mathcal{C} = \{0\}$, since then the decompositions in $\hat{\mathcal{C}}$ are unique. This means that $\hat{\mathcal{C}}$ is a direct sum of $\mathcal{S}$ and $\mathcal{C}$, so $\dim(\hat{\mathcal{C}}) = \dim(\mathcal{S}) + \dim(\mathcal{C})$.

The idea of secret sharing is that the secret stays hidden to some subsets of the participants, while other subsets can reveal the secret. Let $\mathcal{I} \subseteq [n]$ be a subset of the participants. We say that $\mathcal{I}$ is \emph{forbidden} if the encryption scheme given by ciphertext $\hat{c}_\mathcal{I} = s_\mathcal{I} + c_\mathcal{I}$ and message $s$ has perfect security. Further, we say that $\mathcal{I}$ is \emph{admissible} if $\hat{c}_\mathcal{I}$ uniquely determines the secret $s$.

\begin{theorem}
Let $\mathcal{I} \subseteq [n]$. Then, $\mathcal{I}$ is forbidden if and only if $\dim(\hat{\mathcal{C}}_\mathcal{I}) = \dim(\mathcal{C}_\mathcal{I})$. Further, $\mathcal{I}$ is admissible if and only if $\dim(\hat{\mathcal{C}}_\mathcal{I}) = \dim(\mathcal{S}) + \dim(\mathcal{C}_\mathcal{I})$.
\end{theorem}

\begin{proof} \emph{Forbidden sets:}

\ProofLeftarrow Let $\dim(\hat{\mathcal{C}}_\mathcal{I}) = \dim(\mathcal{C}_\mathcal{I})$. Then $\hat{\mathcal{C}}_\mathcal{I} = \mathcal{C}_\mathcal{I}$ and the noise $c_\mathcal{I} \in \mathcal{C}_\mathcal{I}$ is distributed uniformly, so the encryption is a one-time pad. Thus, $\mathcal{I}$ is forbidden.

\ProofRightarrow As $\hat{\mathcal{C}}_\mathcal{I} = \mathcal{S}_\mathcal{I} + \mathcal{C}_\mathcal{I}$, it is clear that $\dim(\hat{\mathcal{C}}_\mathcal{I}) \geq \dim(\mathcal{C}_I)$. Assume that $\dim(\hat{\mathcal{C}}_\mathcal{I}) > \dim(\mathcal{C}_\mathcal{I})$. Then, there is $s \in \mathcal{S}$ and $c \in \mathcal{C}$ such that $s_\mathcal{I} + c_\mathcal{I} \notin \mathcal{C}_\mathcal{I}$ so $s_\mathcal{I} \notin \mathcal{C}_\mathcal{I}$. The ciphertexts of $s_\mathcal{I}$ will therefore not contain $0$, but $0$ is a possible ciphertext for $s = 0$. Thus, the ciphertext $0$ does not have the same probability to be generated for the messages $s$ and $0$. This means that the set $\mathcal{I}$ is not forbidden, as the encryption scheme does not have perfect security.

\emph{Admissible sets:}

\ProofLeftarrow Let $\dim(\hat{\mathcal{C}}_\mathcal{I}) = \dim(\mathcal{S}) + \dim(\mathcal{C}_\mathcal{I})$. Then,
\begin{align*}
    \dim(\hat{\mathcal{C}}_\mathcal{I}) &= \dim(\mathcal{S}_\mathcal{I}) + \dim(\mathcal{C}_\mathcal{I}) - \dim(\mathcal{S}_\mathcal{I} \cap \mathcal{C}_\mathcal{I}) \\
    &\leq \dim(\mathcal{S}) + \dim(\mathcal{C}_\mathcal{I}).
\end{align*}
Therefore, we must have that $\dim(\mathcal{S}_\mathcal{I}) = \dim(\mathcal{S})$ and $\mathcal{S}_\mathcal{I} \cap \mathcal{C}_\mathcal{I} = \{0\}$. Hence, $\hat{c}_\mathcal{I} = s_\mathcal{I} + c_\mathcal{I}$ for unique $s_\mathcal{I} \in \mathcal{S}_\mathcal{I}$ and $c_\mathcal{I} \in \mathcal{C}_\mathcal{I}$. Finally, $s_\mathcal{I}$ uniquely determines $s \in \mathcal{S}$ as the projection to $\mathcal{S}_\mathcal{I}$ is injective.

\ProofRightarrow As $\hat{\mathcal{C}}_\mathcal{I} = \mathcal{S}_\mathcal{I} + \mathcal{C}_\mathcal{I}$, it is clear that $\dim(\hat{\mathcal{C}}_\mathcal{I}) \leq \dim(S) + \dim(\mathcal{C}_\mathcal{I})$. Assume that $\dim(\hat{\mathcal{C}}_\mathcal{I}) < \dim(S) + \dim(\mathcal{C}_\mathcal{I})$. Then, either $\mathcal{S}_\mathcal{I} \cap \mathcal{C}_\mathcal{I} \neq \{0\}$ or $\dim(S_\mathcal{I}) < \dim(S)$. In the first case, there is $0 \neq s \in \mathcal{S}$ and $c \in \mathcal{C}$ such that $s_\mathcal{I} = c_\mathcal{I}$. Then, $\hat{c} = s + 0$ and $\hat{c}' = 0 + c$ have the property that $\hat{c}_\mathcal{I} = s_\mathcal{I} = c_\mathcal{I} = \hat{c}'_\mathcal{I}$, so $s \neq 0$ cannot be uniquely determined from $\hat{c}_\mathcal{I}$. In the second case, there is $0 \neq s \in \mathcal{S}$ such that $s_\mathcal{I} = 0$. Then, $\hat{c} = s + 0$ and $\hat{c}' = 0 + 0$ have the property that $\hat{c}_\mathcal{I} = s_\mathcal{I} = 0 = \hat{c}'_\mathcal{I}$, so $s \neq 0$ cannot be uniquely determined from $\hat{c}_\mathcal{I}$.
\end{proof}

\begin{corollary}\label{cor:dual_distance_and_min_distance_condition}
Let $\mathcal{I} \subseteq [n]$. Then $\mathcal{I}$ is forbidden if $\lvert \mathcal{I} \rvert < d(\mathcal{C}^\perp)$. Further, $\mathcal{I}$ is admissible if $\lvert \mathcal{I} \rvert > n - d(\hat{\mathcal{C}})$.
\end{corollary}

\begin{proof}
If $\lvert \mathcal{I} \rvert < d(\mathcal{C}^\perp)$, then the projection from $\mathcal{C}$ onto $\mathcal{C}_\mathcal{I}$ is surjective by Theorem~\ref{thm:minimum_distance_from_check_matrix}. Hence, $\dim(\mathcal{C}_\mathcal{I}) = \lvert \mathcal{I} \rvert \geq \dim(\hat{\mathcal{C}}_\mathcal{I})$. It is also clear that $\dim(\hat{\mathcal{C}}_\mathcal{I}) \geq \dim(\mathcal{C}_\mathcal{I})$, so $\dim(\hat{\mathcal{C}}_\mathcal{I}) = \dim(\mathcal{C}_\mathcal{I})$. Hence, $\mathcal{I}$ is forbidden.

If $\lvert \mathcal{I} \rvert > n - d(\hat{\mathcal{C}})$, then the projection of $\hat{\mathcal{C}}$ onto $\hat{\mathcal{C}}_\mathcal{I}$ is injective, since otherwise $\hat{\mathcal{C}}$ would contain a nonzero codeword of weight $\leq n - \lvert \mathcal{I} \rvert < d(\hat{\mathcal{C}}_\mathcal{I})$. Therefore,
\begin{equation*}
    \dim(\hat{\mathcal{C}}_\mathcal{I}) = \dim(\hat{\mathcal{C}}) = \dim(\mathcal{S}) + \dim(\mathcal{C}) \geq \dim(\mathcal{S}) + \dim(\mathcal{C}_\mathcal{I}).
\end{equation*}
It is also clear that
\begin{equation*}
    \dim(\hat{\mathcal{C}}_\mathcal{I}) \leq \dim(\mathcal{S}_\mathcal{I}) + \dim(\mathcal{C}_\mathcal{I}) \leq \dim(\mathcal{S}) + \dim(\mathcal{C}_\mathcal{I}),
\end{equation*}
so $\dim(\hat{\mathcal{C}}_\mathcal{I}) = \dim(\mathcal{S}) + \dim(\mathcal{C}_\mathcal{I})$. Hence, $\mathcal{I}$ is admissible.
\end{proof}

\begin{example}
In Shamir secret sharing we choose $\mathcal{S} = \{(\lambda, \dots, \lambda) \mid \lambda \in \F_q \}$ to be the repetition code. Further, $\mathcal{C} = \{ \ev(x \cdot p) \mid p \in \F_q[x], \deg(p) < t \}$, where $\ev \colon \F_q[x] \to \F_q^n$ denotes the evaluation map on the points $\alpha_1, \dots, \alpha_n$. We may also write $\mathcal{S} = \RS_1(\alpha)$ and $\mathcal{C} = \alpha \star \RS_t(\alpha) = \GRS_t(\alpha, \alpha)$. Then, $\hat{\mathcal{C}} = \mathcal{S} + \mathcal{C} = \RS_{t + 1}(\alpha)$. By Corollary~\ref{cor:dual_distance_and_min_distance_condition} any set of size $\leq d(\mathcal{C}^\perp) - 1 = (t + 1) - 1 = t$ is forbidden, while any set of size $> n - d(\hat{\mathcal{C}}) = n - (n - (t + 1) - 1) = t$ is admissible.
\end{example}

\begin{remark}\label{rmk:trivial_intersection_from_weight}
Let $\mathcal{C} \subseteq \F_q^n$ be a linear code. As we want to find $\mathcal{S} \subseteq \F_q^n$ such that $\mathcal{S} \cap \mathcal{C} = \{0\}$, we may choose $\mathcal{S} = \spn \{s\}$, where $s \neq 0$ and $\wt(s) \leq d(\mathcal{C}) - 1$. This makes $\mathcal{S}$ a one-dimensional code that is guaranteed to intersect $\mathcal{C}$ trivially. In general, if $\mathcal{S}$ is supported on some $\leq d(\mathcal{C}) - 1$ coordinates, then $\mathcal{S} \cap \mathcal{C} = \{0\}$.
\end{remark}

\subsection{Homomorphic secret sharing}

In many applications of secret sharing we are interested in computing with data instead of just storing and retrieving data. In particular, given the secret shares of some secrets we want to be able to compute the secret share of a function of these secrets. This type of secret sharing is known as \emph{homomorphic secret sharing}.

\begin{example}
Let $s^{(1)}, s^{(2)} \in \F_q$ denote two secret values and $\hat{f}^{(1)}, \hat{f}^{(2)}$ be their Shamir secret sharing polynomials of degree $\leq t$ whose evaluations are shared to the participants. Therefore, $(\hat{f}^{(1)} + \hat{f}^{(2)})(0) = s^{(1)} + s^{(2)}$ and $(\hat{f}^{(1)} \hat{f}^{(2)})(0) = s^{(1)} s^{(2)}$. The participants can compute
\begin{equation*}
    (\hat{f}^{(1)} + \hat{f}^{(2)})(\alpha_i) = \hat{f}^{(1)}(\alpha_i) + \hat{f}^{(2)}(\alpha_i)
\end{equation*}
and
\begin{equation*}
    (\hat{f}^{(1)} \hat{f}^{(2)})(\alpha_i) = \hat{f}^{(1)}(\alpha_i) \hat{f}^{(2)}(\alpha_i).
\end{equation*}
from their local data. Therefore, we may interpolate the polynomials $\hat{f}^{(1)} + \hat{f}^{(2)}$ (degree $\leq t$) and $\hat{f}^{(1)}\hat{f}^{(2)}$ (degree $\leq 2t$) from any $t + 1$ and $2t + 1$ shares, respectively.
\end{example}

Let $\hat{c}^{(1)} = s^{(1)} + c^{(1)}$ and $\hat{c}^{(2)} = s^{(2)} + c^{(2)}$ denote two secret shares. Then, $\hat{c}^{(1)} + \hat{c}^{(2)} = (s^{(1)} + s^{(2)}) + (c^{(1)} + c^{(2)})$ is a secret share of the sum of the secrets $s^{(1)} + s^{(2)} \in \mathcal{S}$. Indeed, the noise term $c^{(1)} + c^{(2)} \in \mathcal{C}$ will also be uniformly distributed. Furthermore, the participants can compute their shares using just their local data as $\hat{c}^{(1)}_i + \hat{c}^{(2)}_i$.

Scalar multiplications of the secret by a public scalar $\lambda \in \F_q$ are similarly easy. In particular, $\lambda \hat{c} = \lambda s + \lambda c$ is a secret share of $\lambda s$, where the noise term $\lambda c \in \mathcal{C}$ is uniformly distributed provided that $\lambda \neq 0$. This means that any linear function of the secret shares can easily be computed from the shares held by the participants.

Instead of scalar multiplication, we may consider multiplication by another (public) vector. In particular, let $\mathcal{E} = \spn\{e\}$ for some $e \neq 0$ and secret share it with $\hat{d} = e + d$, where $d \in \mathcal{D}$ is chosen uniformly at random. If $c \in \mathcal{C}$, then $c \star \hat{d} = c \star e + c \star d$, which looks a bit like the secret share of $c \star e$, where the noise is drawn from the space $\mathcal{C} \star \mathcal{D}$. However, $c \star d \in \mathcal{C} \star \mathcal{D}$ is not distributed uniformly at random in the subspace. Nevertheless, if $\wt(e) \leq d(\mathcal{C} \star \mathcal{D}) - 1$, then $c \star e$ can be uniquely decoded from $c \star \hat{d}$ as $(\mathcal{C} \star \mathcal{E}) \cap (\mathcal{C} \star \mathcal{D}) = \{0\}$. This follows directly from Remark~\ref{rmk:trivial_intersection_from_weight}.

If we have two secret shares coming from the (potentially different) codes $\mathcal{S}^{(1)}, \mathcal{S}^{(2)}$ and $\mathcal{C}^{(1)}, \mathcal{C}^{(2)}$. Then the star product of the shares will be
\begin{equation*}
    (s^{(1)} + c^{(1)}) \star (s^{(2)} + c^{(2)}) = s^{(1)} \star s^{(2)} + s^{(1)} \star c^{(2)} + c^{(1)} \star s^{(2)} + c^{(1)} \star c^{(2)}.
\end{equation*}
The first term here contains the information we care about, namely $s^{(1)} \star s^{(2)}$, while the last three terms contain some noise. In this case it is a bit more difficult to control the intersection between the defining codes of this secret sharing scheme. In particular, the dimension of the space $\mathcal{C}^{(1)} \star \mathcal{C}^{(2)}$ may be very large for generic codes $\mathcal{C}^{(1)}$ and $\mathcal{C}^{(2)}$.

\end{document}